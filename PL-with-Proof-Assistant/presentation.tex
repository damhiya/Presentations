\documentclass{beamer}
\usetheme{metropolis}
\usecolortheme{seahorse}

\usepackage{kotex, minted, csquotes}

\usepackage{fontspec}
\setsansfont{Noto Sans}
\setsanshangulfont{Noto Sans CJK KR}
\setmonofont{Iosevka Fixed Slab}

\usepackage{unicode-math}
\setmathfont{STIX Two Math}
\usepackage{tikz-cd}

\renewcommand{\indent}{\hspace*{2em}}
\newcommand\Beh[0]{\operatorname{Beh}}

\title{Formalization of Programming Languages}
\subtitle{Define semantics of PL in proof assistant}
\author{문순원}
\date{}

\begin{document}

\begin{frame}
  \titlepage
\end{frame}

\begin{frame}{증명언어 소개}

  \begin{block} {형식증명 (formal proof)}
    \begin{itemize}
      \item 증명을 형식적으로 표현
      \item \( \leftrightarrow \) 자연어 증명
      \item \textbf{기계로 검증할 수 있다}
    \end{itemize}
  \end{block}

  \pause
  \begin{block} {증명보조기 (proof assistant)}
    \begin{itemize}
      \item 형식증명의 작성을 도와주는 소프트웨어
    \end{itemize}
  \end{block}

  \pause
  \begin{block} {증명언어 (proof language)}
    \begin{itemize}
      \item 형식증명을 표현할 수 있는 프로그래밍 언어
      \item Agda, Coq, Isabelle, Lean 등
    \end{itemize}
  \end{block}

\end{frame}

\begin{frame}{증명언어의 특징}

  \begin{block} {Propositions as types}
    \begin{itemize}
      \item 증명을 수학적 대상으로 다룬다
      \item 명제는 프로그램의 타입에 대응
      \item 명제의 증명은 프로그램에 대응
    \end{itemize}
  \end{block}

  \pause
  \begin{block} {강타입 함수형 언어와 공유하는 특징 (e.g. Haskell, OCaml)}
    \begin{itemize}
      \item Purely functional
      \item Strongly typed
    \end{itemize}
  \end{block}

  \pause
  \begin{block} {공유하지 않는 특징}
    \begin{itemize}
      \item Total functional
      \item Dependent type
    \end{itemize}
  \end{block}

\end{frame}

\begin{frame}{증명언어와 관련된 분야}

  \begin{block}{증명언어에 대한 연구}
    \begin{itemize}
      \item 수학, PL 분야의 공통적 연구대상
      \item 타입이론
    \end{itemize}
  \end{block}

  \pause
  \begin{block}{증명언어를 활용}
    \begin{itemize}
      \item 수학 증명 형식화 \alert<3>{(e.g. Four color theorem)}
      \item 프로그래밍 언어 형식화
    \end{itemize}
  \end{block}

  \begin{block}<4>{PL 형식화의 필요성}
    \begin{itemize}
      \item 프로그래밍 언어 설계상의 결함을 제거
      \item 프로그램을 검증하기 위한 밑바탕
    \end{itemize}
  \end{block}
\end{frame}

\begin{frame}{증명언어로 검증된 프로그램을 만드는 방법}

  \begin{block}{프로그램을 증명언어로 작성}
    \begin{itemize}
      \item 증명언어를 OCaml, Haskell, Scheme, JavaScript 등의 언어로 컴파일
      \item 비교적 쉽다
    \end{itemize}
  \end{block}

  \pause
  \begin{block}{프로그램을 다른 언어로 작성}
    \begin{itemize}
      \item C 언어를 증명언어에서 형식화
      \item 형식화된 대상언어로 작성된 프로그램을 증명언어로 검증
      \item 매우 어렵다
    \end{itemize}
  \end{block}

  \pause
  \begin{block}{예시}
    \begin{itemize}
      \item CompCert : Coq
      \item CertiKOS : C, Coq
      \item seL4 : C, Isabelle
    \end{itemize}
  \end{block}

\end{frame}

\begin{frame}{프로그래밍 언어의 의미론}

  \begin{block}{Denotational semantics}
    \begin{itemize}
      \item 프로그램을 수학적 대상에 대응 시킨다
      \item 주로 FP언어에서 많이 사용
    \end{itemize}
  \end{block}
  
  \pause
  \begin{block}{Operational semantics}
    \begin{itemize}
      \item 추상기계의 상태 전이를 통해 의미를 부여
      \item 명령형 언어에서 흔히 사용되는 접근
    \end{itemize}
  \end{block}

  \pause
  \begin{block}{Axiomatic semantics}
    \begin{itemize}
      \item 프로그램에 대한 공리들을 통해 프로그램을 검증
      \item Hoare logic 등
    \end{itemize}
  \end{block}

  \pause
  \footnotesize Operational semantics를 통해 Hoare logic의 공리들을 증명하는 것도 가능

\end{frame}

\begin{frame}{프로그래밍 언어의 의미론}

  \begin{block}{Program logic}
    \begin{displayquote}
      \footnotesize
      Verifiable C is based on a 21st-century version of Hoare logic called higher-order impredicative concurrent separation logic. Back in the 20th century, computer scientists discovered that Hoare Logic was not very good at verifying programs with pointer data structures; so separation logic was developed. Hoare Logic was clumsy at verifying concurrent programs, so concurrent separation logic was developed. Hoare Logic could not handle higher-order object-oriented programming patterns or function-closures, so higher-order impredicative program logics were developed.
      \par\raggedleft--- Preface, Software Foundations volume 5
    \end{displayquote}

    \begin{itemize}
      \item Hoare logic
      \item separation logic
      \item concurrent separation logic
      \item higher-order impredicative program logics
    \end{itemize}
  \end{block}

\end{frame}

\begin{frame}{프로그래밍 언어의 의미론}

  \begin{block}{Behavioral refinement}
    \[ \Beh(T) \subseteq \Beh(S) \]
    \begin{itemize}
      \item source \(S\)를 target \(T\)로 변환하는 것이 옳바른가?
      \item \footnotesize Compiler : \(S\) = 프로그래머가 작성한 소스코드, \(T\) = 컴파일러의 출력물 \normalsize
      \item \footnotesize Software verification : \(S\) = 추상화된 스펙, \(T\) = 검증 대상 프로그램
    \end{itemize}
  \end{block}

\end{frame}

\begin{frame}{프로그래밍 언어의 의미론}

  \[ \Beh(T) \subseteq \Beh(S) \]

  \footnotesize
  \begin{block}{undefined behavior}
    \begin{itemize}
      \item 전체 집합
      \item \(S\)에서 UB가 발생할 경우, 임의의 \(T\)에 대해 refinement가 성립
      \item \(T\)에서 UB가 발생할 경우, \(S\)에서 UB가 발생해야 refinement가 성립
    \end{itemize}
  \end{block}

  \pause
  \begin{block}{no behavior}
    \begin{itemize}
      \item 공집합
      \item \(T\)에서 NB가 발생할 경우, 임의의 \(S\)에 대해 refinement가 성립
      \item \(S\)에서 NB가 발생할 경우, \(T\)에서 NB가 발생해야 refinement가 성립
    \end{itemize}
  \end{block}

\end{frame}

\begin{frame}{프로그래밍 언어의 의미론}

  \begin{block} {Compiler}
    \begin{itemize}
      \item \(T\)에서 NB를 발생시킬 경우 컴파일이 자명하게 옳바름
      \item target에서 NB를 발생시키는 것을 금지해야 한다
    \end{itemize}
  \end{block}

  \pause
  \begin{block} {Software verification}
    \begin{itemize}
      \item \(S\)에서 UB를 발생시킬 경우 증명 대상이 스펙을 자명하게 만족
      \item source에서 UB를 발생시키는 것을 금지해야 한다
    \end{itemize}
  \end{block}

  \pause
  \[ \Beh(\text{Asm}) \subseteq \Beh(\text{Imp}) \subseteq \Beh(\text{Spec}) \]
\end{frame}

\begin{frame}[fragile]{프로그래밍 언어의 의미론}

  \begin{block}{Simulation}
    \[
      \begin{tikzcd}
        e_{\text{src}} \arrow{r}{q} \arrow[dash]{d} & e_{\text{src}}' \arrow[dash]{d} \\
        e_{\text{tgt}} \arrow[dashed]{r}{q} & e_{\text{tgt}}'
      \end{tikzcd}
    \]
    
    \begin{itemize}
      \item labeled state transition system 사이의 preorder
      \item Behavioral refinement를 증명하기 위한 중간 단계로 자주 사용됨
    \end{itemize}
  \end{block}

  \pause
  \begin{block}{개인적인 감상}
    \begin{itemize}
      \item 프로그램의 behavior는 event 만으로 정의됨
      \item event에 대해 논증하기 위해 state가 필요함
    \end{itemize}
  \end{block}

\end{frame}

\begin{frame}{Program logic vs Behavioral refinement}

  \begin{block}{Program logic}
    \begin{itemize}
      \item VST (Verified Software Toolchain)
      \item Iris
      \item RefinedC
    \end{itemize}
  \end{block}

  \begin{block}{Behavioral refinement}
    \begin{itemize}
      \item CompCert
      \item CertiKOS
    \end{itemize}
  \end{block}

\end{frame}

\end{document}
